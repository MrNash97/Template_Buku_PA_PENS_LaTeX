% ******************************* Thesis Dedication ********************************

\begin{dedication} 
\begin{changemargin}{5mm}{-5mm}
\newcounter{nameOfYourChoice}
Dengan menyebut nama Allah Yang Maha Pengasih lagi Maha Penyayang, penulis menyelesaikan buku proyek akhir dengan judul:

\begin{center}
\textbf{\textit{AUTONOMOUS DRIFTING MOBILE ROBOT:}}
\textbf{Kontrol Kemudi Dua Roda untuk Meminimalisir \textit{Oversteer} dan \textit{Understeer} pada \textit{Mobile Robot}}
\end{center}

Proyek akhir ini merupakan salah satu syarat yang harus dipenuhi untuk meyelesaikan program studi Diploma 4 pada Jurusan Teknik Mekatronika Politeknik Elektronika Negeri Surabaya. Melalui kegiatan ini diharapkan mahasiswa dapat melakukan kegiatan laporan yang bersifat penelitian ilmiah dan menghubungkannya dengan teori yang telah diperoleh dalam perkuliahan. Buku ini juga disusun sepenuh hati dengan harapan pembaca mendapatkan ilmu dan gambaran dengan jelas tentang apa yang penulis kerjakan.

Pada kesempatan ini penulis panjatkan puji syukur kehadirat Allah SWT atas segala nikmat yang telah diberikan-Nya. Shalawat serta salam tidak lupa kita curahkan kepada junjungan nabi besar Nabi Muhammad SAW beserta keluarga, para sahabat dan umatnya hingga akhir zaman. Serta tidak lupa ucapan terima kasih yang sebesar-besarnya kepada beberapa pihak yang telah memberikan dukungan selama proses penyelesaian proyek akhir ini, antara lain:

\begin{enumerate}
    \item Orang tua penulis, Ayah \textbf{Harris Purwanto} dan Ibuk \textbf{Aries Surjani Edowati}, saudara penulis \textbf{Mas Gugik, Tita, dan Ruri}, serta seluruh keluarga besar yang selalu mengalirkan doa, memberikan semangat, nasehat, pengertian, dan dengan penuh kesabaran dalam membimbing penulis.
    \item Bapak \textbf{Raden Sanggar Dewanto, S.T., M.T., Ph.D.} dan \textbf{Eko Henfri Binugroho, S.ST., M.Sc.} selaku dosen pembimbing proyek akhir yang telah banyak membantu dan membimbing hingga laporan ini dapat terselesaikan.
    \item Bapak \textbf{Didik Setyo Purnomo, S.T., M.Eng.} selaku kepala Departemen Teknik Mekanika dan Energi
    \item Ibu \textbf{Endah Suryawati Ningrum, S.T., M.T.} selaku Kepala Program Studi Teknik Mekatronika PENS.
    \setcounter{nameOfYourChoice}{\value{enumi}}
\end{enumerate}\end{changemargin}
    \chapter*{}
    \addtocounter{page}{-1}
    \thispagestyle{empty} 
    \NewPage
    \begin{enumerate}
    \setcounter{enumi}{\value{nameOfYourChoice}}
    \item Bapak dan Ibu dosen penguji proyek akhir yang telah memberikan saran dan masukannya kepada penulis.
    \item Semua Bapak Ibu dosen di lingkugan PENS khususnya Jurusan Teknik Mekatronika yang telah memberikan ilmu, nasehat, serta waktunya dengan ikhlas selama ini.
    \item \textbf{Pak Gatut, Mas Akbar, dan Mas Yasin} yang telah membagikan ilmunya.
    \item \textbf{Ajis} dan \textbf{Charis} yang berjuang bersama demi mimpinya masing-masing.
    \item Teman-teman Kantor Genjer-211 \textbf{Meri, Nurul, Ica, Fay, Wildan, Pau, Roi, Dhanis, Edy} dan \textbf{Om} yang senantiasa bercanda tawa ceria bersama penulis.
    \item Keluarga besar \textbf{BADAN EKSEKUTIF MEKATRONIKA} yang telah tumbuh bersama selama empat tahun.
    \item Teman-teman \textbf{D4 Teknik Mekatronika 2014} yang luar biasa dan selalu saling memberi support.
    \item \textbf{Mas Arfaq, Mas Chrisna, Mas Sholeh, Mas Wanda}, alumni bengkel yang juga sangat sangat membantu penulis.
    \item \textbf{Uwik} dan \textbf{Siti}, yang tak pernah lelah menguatkan penulis untuk tetap sabar, semangat, dan bersyukur.
    \item \textbf{SMIFER}! \textbf{Milla, Ilham, Fajri, Endutbilqis, dan Re}, yang juga sedang berjuang tuk mewujudkan mimpinya masing-masing.
    \item Calon Mama Cantik, \textbf{Andin} \& \textbf{Intan}.
    \item Keluarga besar \textbf{BEM PENS Periode 2015/2016 dan 2016/2017} yang telah mengajarkan penulis untuk lebih berkembang dan berkarya.
    \item Serta semua pihak yang telah membantu kelancaran pelaksanaan proyek akhir yang tidak bisa disebutkan satu persatu.
\end{enumerate} 

Dalam penyusunan laporan proyek akhir ini penulis menyadari akan adanya kekurangan-kekurangan baik dalam penyususnan maupun pembahasan masalah karena keterbatasan pengetahuan penulis. Untuk itu penulis mengharapkan kritik dan saran membangun dari semua pihak agar dapat lebih baik di masa yang akan datang. Aamiin. Terima kasih.

\addcontentsline{toc}{chapter}{Kata Pengantar}
\end{dedication}

