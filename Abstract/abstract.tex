% ************************** Thesis Abstract *****************************
% Use `abstract' as an option in the document class to print only the titlepage and the abstract.
\begin{abstract}
%	\begin{changemargin}{5mm}{-5mm}

%Kesehatan merupakan hal yang vital bagi setiap individu. Melakukan pengecekan kesehatan secara berkala serta antisipasi penyakit sejak dini merupakan kebutuhan yang seharunya dilakukan oleh setiap orang supaya mendapatkan rasa aman karena terhindar dari penyakit serta mendapatkan penanganan yang cepat dan tepat saat terindikasi mengidap penyakit tertentu. Berdasarkan penelitian \citet{fieselmann1993}, \textit{Breathing Rate} (BR) adalah salah satu indikator tanda vital utama, dan sering digunakan untuk menyimpulkan status kesehatan kardiopulmonari subjek. Sebagai contoh, tingkat pernapasan yang lebih tinggi dari 27 kali per menit adalah prediktor paling penting untuk pasien serangan jantung. Metode yang paling umum digunakan untuk mengukur HR dan BR adalah menghitung secara manual, selain itu HR dan BR dapat diukur menggunakan berbagai sensor yang dipasangkan pada tubuh. Namun, penggunaan sensor-sensor untuk monitor HR dan BR dalam praktisnya sulit untuk diterapkan dalam kondisi tertentu karena pasien cenderung merasa tidak nyaman, misalnya pada bayi yang aktif bergerak atau ketika gerakan bebas diperlukan, diagnosa luka (luka bakar / ulkus / trauma) dan evaluasi penyembuhan kulit. Oleh karena itu, penelitian ini bertujuan untuk merancang sebuah sistem yang dapat mengukur tingkat HR dan BR tanpa diperlukan kontak langsung dengan tubuh dengan menggunakan teknik \textit{Photoplethysmography} (PPG) yang memanfaatkan pengolahan citra dan pengolahan sinyal dengan  menggunakan kerangka kerja \textit{Eulerian Video Magnification (EMV)}.

Dalam proses monitoring detak jantung atau \textit{Heart Rate} (HR) dan pernapasan atau \textit{Breath Rate} (BR) konvensional penggunaan sensor-sensor yang dipasangkan pada tubuh dalam praktisnya sulit untuk diterapkan dalam kondisi tertentu karena pasien cenderung merasa tidak nyaman, misalnya pada bayi yang aktif bergerak atau ketika gerakan bebas diperlukan, diagnosa luka (luka bakar / ulkus / trauma) dan evaluasi penyembuhan kulit, ditambah lagi penempatan posisi sensor pada tubuh dapat mempengaruhi tingkat akurasi pembacaan. Oleh karena itu, Pada proyek akhir ini dibuat suatu sistem berbasis sensor kamera yang dapat mengukur tingkat HR dan BR tanpa adanya kontak fisik dengan tubuh dengan menggunakan teknik \textit{Photoplethysmography} (PPG) yang memanfaatkan pengolahan citra dan pengolahan sinyal dengan menggunakan kerangka kerja \textit{Eulerian Video Magnification} (EVM). Tujuan akhir untuk memonitoring kinerja jantung hanya melalui visual. Hasil pengujian sementara dalam penguatan informasi gerak bekerja dengan baik pada daerah leher, sedangkan penguatan informasi warna yang dapat merepresentasikan detak jantung ditemukan pada daerah wajah. Ekstraksi sinyal dari setiap kanal warna RGB \textit{(Red-Green-Blue)} dilakukan untuk memperoleh data detak jantung.

{\noindent \textbf{Kata Kunci:} \textit{Kesehatan, Non-Kontak Monitoring, \textit{Photoplethysmography} (PPG), Pengolahan Citra, Pengolahan Sinyal, \textit{Eulerian Video Magnification} (EVM).}}
\addcontentsline{toc}{chapter}{ABSTRAK}
%\end{changemargin}
\end{abstract}
