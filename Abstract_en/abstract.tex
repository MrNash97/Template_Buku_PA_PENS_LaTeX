% ************************** Thesis Abstract *****************************
% Use `abstract' as an option in the document class to print only the titlepage and the abstract.

\begin{abstract_en}
%\textit{Having a healthy body is an important matter for everyone. Perform a regular medical check up and early disease prevention is a necessity that should be done by each individual in order to get assurance and sense of security from a disease and also get a fast and precise treatment when indicated suffering from a certain diseases. Based on \citet{fieselmann1993}, Breathing Rate (BR) is one of the main vital sign indicators and is often used to infer subject's cardiopulmonary status. It's also reported that a respiratory rate of 27 or more was the most important predictor of cardiac arrest in hospital wards. The most commonly used method of measuring HR and BR is by manual calculaiton, in addition, HR and BR can be measured using various sensors attached to the body. However, the use of sensors for HR and BR monitoring in practice is difficult to apply under certain circumstances because the patient is likely to feel uncomfortable, for example in an active infant  or when free movement required, wound diagnosis (burn / ulcers / trauma) and evaluation of skin healing. Therefore, this study aims to design a system that can measure the level of HR and BR without direct contact with the body using \textit{Photoplethysmography} (PPG) which utilizes image processing and signal processing using \textit{Eulerian Video Magnification} (EMV) frameworks.}

\textit{In the conventional \textit{Heart Rate} (HR) and \textit{Breath Rate} (BR) monitoring process the use of sensors attached to the body is practically difficult to apply under certain conditions because patients tend to feel uncomfortable, for example in infants who are actively moving or when free movement is needed, diagnosing wounds (burns / ulcers / trauma) and evaluation of skin healing, plus the placement of the sensor position in the body can affect the accuracy of the reading. Therefore, in this final project a camera sensor based system can be used to measure HR and BR levels without physical contact with the body using the Photoplethysmography (PPG) technique that utilizes image processing and signal processing using the Eulerian Video Magnification (EVM) framework. The ultimate goal of this study is to monitor heart conditions and respiratory system only through visual. Tentative results in motion magnification work well in the neck area, while color amplification that can represent heart rate (HR) is found in the facial area.~Signal extraction from each RGB (Red-Green-Blue) color channel is performed to obtain heart rate data.}

{\noindent \textit{\textbf{Keywords:} Health, Non-wearable Monitoring, Photoplethysmography (PPG), Image Processing, Signal Processing, Eulerian Video Magnification (EVM).}}
\addcontentsline{toc}{chapter}{\textit{ABSTRACT}}
\end{abstract_en}
